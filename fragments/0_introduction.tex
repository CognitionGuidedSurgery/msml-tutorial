\section{Introduction}
\label{sec:introduction}

We first mentioned the idea from anunified and mighty workflow system for biomechanical engineering  in
\cite{Suwelack:2014}. In the meantime we realize the Medical
Simulation Markup Language (MSML).

You can see MSML as a build tool, like \emph{GNU Make}, except the
offered operations are settled in the pre- and postprocessing of
three dimensional data after the biomechanical simulation.
The simulation is the heart of every workflow and is outsourced to
modern simulation frameworks, like
SOFA\footnote{http://www.sofa-framework.org/} or
Hiflow3\footnote{http://hiflow3.org}, and MSML cares about the messy
details of the simulation framework. You get an unified interface.

\paragraph{Content}
\label{sec:content}

In this paper we show the various power of MSML in different
scenarios (sections~\ref{sec:tut1,sec:tut2,sec:tut3}).
Before the tutorials we create a view on MSML in
section~\ref{sec:msml-architecture}. This chapter defines the ground
terms and nomenclatures in MSML. Openness is one power of MSML. You
can extend it in various kinds, such topics are discovered in
section~\ref{sec:extend}.

\paragraph*{Acknowledgement}
\label{sec:acknowledgement}

MSML was developed in the SFB\,TRR\,125 Cognition Guided Surgery
project, funded by the German Research Foundation (DFG).


%%% Local Variables:
%%% mode: latex
%%% TeX-master: "../tutorial"
%%% End:
