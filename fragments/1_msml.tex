\section{MSML Architecture}
\label{sec:msml-architecture}

We start with the terms, give an overview of the pipeline and end this
section with the MSMLFile datastructure.

\paragraph{Terms}

An \emph{Operator} is a function, that can be used within the
workflow. The function can be written in C/C++ or Python. We maintain additional
meta data of the arguments and output in a separate XML
file. If an Operator is used, it get instantiated with the given
references to other instances or variables. We call instances of
Operator a \emph{Task}. \emph{Elements} define additional features,
like materials or constraints, to your objects in the scene.
The \emph{Alphabet} contains both, Elements and Operators, with their
XML meta data. You can add new Operators or Elements to MSML by
creating new Alphabet entries. The simulation is outsourced to several
simulation frameworks. The interface between MSML and the frameworks
is done within several specific \emph{Exporter}. Currently we support
Sofa, Hiflow, FeBio and Abaqus.

\paragraph{Pipeline}
\label{sec:pipeline}

Figure~\ref{fig:pipeline} gives an overview about the process.
MSML takes a data structure as shown in figure~\ref{fig:model}.
The data structure can allocated from XML or with Python.
Defining in Python offers more control over the pipeline and is more
flexible, as you can use parameterized the creation.
XML is easy, understandable but more fixed.

The first step analyze the data structure for finding errors and
creating a build graph. The build graph is a directed acyclic
graph\footnote{You mustn't have cycles.}.
The generated build graph is checked for consistency of types.
We try to solve type errors by injecting conversions in the graph.
Such a graph is in figure~\ref{fig:bunnygraph}.

An \emph{Executor} takes control over the  pre-, postprocessing and
simulation. It executes the build graph in the correct
order. Currently We are offering three Executors:

\begin{description}
\item[LinearSequence] is a simple execution of the task in topological order.
\item[Parallel] executes a set of independently task in process or
  thread pool.
\item[Phase] gives you control, to disable certain phases, task or
  exporter.
\end{description}

The distinction between the three phases is a logical separation. The
build graph only knows nodes, that are tasks, variables or exporter.

\begin{figure}
  \centering
  \tikzset{
    pp state/.style = {draw, black, text width=7em, text height=1.2em,
      rounded corners, text centered, text depth=.25ex},
    pp line/.style = {->,draw, line width=0.8 pt},
    pp border/.style = {dotted, line width=1pt}
  }

  \begin{tikzpicture}[node distance=0.5cm]

    \node[pp state] (model) {MSML Model};

    \node[pp state, above right=0.6cm and -1.4cm of model] (XML) {XML file};
    \node[pp state, above left= 0.6cm and -1.4cm  of model] (py) {Python};

    \node[pp state, below=of model] (bg)    {Build Graph};

    \node[pp state, below=of bg] (pre) {Preprocessing};
    \node[pp state, below=of pre] (sim) {Simulation};
    \node[pp state, below=of sim] (post) {Postprocessing};

    \path[pp line]
      (py) edge (model)
      (XML) edge (model)
      (model) edge (bg)
      (bg) edge (pre)
      (pre) edge (sim)
      (sim) edge (post)
    ;


    \node[rotate=90] at ($(pre.north west) + (-0.5,-0.6)$) (E) {Executor};

    \draw[pp border]
      ($(pre.north west) + (-0.3,0.3)$) rectangle
      ($(post.south east) + (+0.3,-0.3)$)
    ;

  \end{tikzpicture}

  \caption{MSML Pipeline}
  \label{fig:pipeline}
\end{figure}


\paragraph{Data structure}
\label{sec:data-structure}


The \emph{MSML File} data structure is the central interface between
you and MSML. It gathers all information for making a simulation:

\begin{description}
\item[Workflow] holds the list of Tasks to be executed
\item[Environment] is the set of parameters and configuration for the
  simulation system.
\item[Variables] are reusable placeholders, that allow to create
  parameterized workflows.
\item[Scene Object] encapsulates an object in the simulation, that can
  be enriched by material regions, constraints or output requests.
\end{description}

The structure is visible in the first example~\ref{fig:bunnymsml} and
figure~\ref{fig:model} shows the cardinality.


\begin{figure*}
  \centering
  \tikzset{
    uml class/.style = {draw, black, minimum width = 8em, minimum
      height = 2em, text centered, text depth=.25ex, rectangle},
    uml class 3/.style = {draw, black, minimum width = 8em, minimum
      height = 2em, text centered, text depth=.25ex, rectangle split,
      rectangle split parts=3},
    uml comp/.style ={>=diamond}
  }
  \begin{tikzpicture}
    %\draw[ultra thin,gray] (-10,-10) grid (10,10);

    \node[uml class] (MF) {MSMLFile};

    \node[uml class,left=of MF] (SO) {SceneObject};

    \node[uml class,right=of MF] (Env) {Environment};

    \node[uml class,above=of MF] (WF) {Workflow};

    \node[uml class,below=of MF] (V) {Variables};

    \node[uml class,above=of WF] (T) {Task};

    \node[uml class,above left=of SO] (C) {Constraints};
    \node[uml class,left=of SO] (MR) {Material Region};
    \node[uml class,below left=of SO] (OR) {Output};

    \newcommand\compositon[2]{
      \draw [{Diamond}-{Stealth}] #1 -- #2;
    }

    %\draw[{Diamond}-{Stealth}] (V) --
    %  node[near end, right]   {1}
    %  node[near start, right] {*}
    %  (MF);


    \umlunicompo[mult=*]{MF}{V}
    \umluniassoc[anchors=east and west]{MF}{Env}
    \umluniassoc{MF}{WF}
    \umluniassoc[mult1=1,mult2=*]{MF}{SO}

    \umlunicompo[mult=]{SO}{C}
    \umlunicompo[mult=]{SO}{OR}
    \umlunicompo[mult=]{SO}{MR}

    \umlunicompo[mult=*]{WF}{T}

  \end{tikzpicture}
  \caption{MSML Model}
  \label{fig:model}
\end{figure*}

\paragraph{Sort logic}
\label{sec:sort-logic}



%%% Local Variables:
%%% mode: pdflatex
%%% TeX-master: "../tutorial"
%%% End:
