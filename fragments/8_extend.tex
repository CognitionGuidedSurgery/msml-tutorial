\section{Extend MSML}
\label{sec:extend}

MSML is a platform, that can be extended in multiple fashions.
The terms explain in section~\ref{sec:msml-architecture}

\subsection{Operator}
\label{sec:operator}

An operator in MSML is a function that perform an operation. At least
the function has to executable within Python. For C/C++ we provide a
CMAKE build environment and wrapping with SWIG. Additional we provide
operator adapters for external programs or shared object (ctype).

\begin{figure}
  \centering
\begin{lstlisting}[language=Python]
import vtk
def cp(mesh, ref):
  locator = vtk.vtkPointLocator()
  ugrid = read_ugrid(mesh)
  locator.SetDataSet(ugrid)

  index = locator.FindClosestPoint(ref)
  point = ugrid.GetPoint(index)
  distance = distance(ref, point)
  return {'index': index,
          'point': point,
           'dist': distance}
\end{lstlisting}
  \caption{Python snippet of a function for calculating the nearest
    point in \texttt{mesh} from \texttt{ref}}
  \label{fig:get_points}
\end{figure}

\begin{figure*}[h]
  \centering
  \lstinputlisting[language=XML]{assets/operatorexample.xml}
  \caption{Operator Definiton Example}
  \label{fig:operatorxml}
\end{figure*}


Let's assume we want to provide
\lstinline[language=Python]{cp(mesh, ref)} from
figure~\ref{fig:get_points} in MSML. The function calculates the
closest point in \lstinline[language=Python]{mesh} to a given
reference vector \lstinline[language=Python]{ref}.
Save the function in Python module, make sure MSML can import this
module by setting \lstinline[language=Python]{PYTHONPATH} or using the
\lstinline[language=Python]{--operator-dir} on the command line.
The next step is to create the entry in the
Alphabet. Figure~\ref{fig:operatorxml} shows an accuarate one.
The runtime gives the type of the operator. For a Python Operator you
need to specify the module and the function name. Input, output and
parameters contains list of args. The order of args determines the
order in which the arguments are given in the function call. Here we
define one input argument, that should be given as a VTK object, and
one parameter as a list of floats values. This operators delivers
three different output values.
Once you added the XML file (figure~\ref{fig:operatorxml}) to the
alphabet search path or add \lstinline{--alphabet-search-dir} on the
command line, you should be able to call this operator within MSML
\lstinline[language=XML]%
{<closestPoint id="c"  mesh="${mesh}"  ref="2.2 3.5 6"/>}

You can access to every output with
\lstinline{c.distance}, \lstinline{c.point} and \lstinline{c.index}.

\subsection{Element}
\label{sec:element}

An Element defines the semenatics of an entry under constraint,
material region or output. They should be a disambiguous description
of their use. Every exporter handles the elements on his own. An
introduction of an element leads to an adaption of the exporters.
The Exporter features determines the supported elements.
Figure~\ref{fig:elementxml} shows an element definition. It consists
from a name, category, description and parameters. The definition of
parameters are the same as for operators.

\begin{figure}
  \centering
  \lstinputlisting[language=XML]{assets/elementexample.xml}
  \caption{Operator Definiton Example}
  \label{fig:elementxml}
\end{figure}

\subsection{Exporter}
\label{sec:exporter}

The Exporter are a central point of the system. Their task is to
create the input for the specific simulation environment, run the
simulation and provide the calculated output back into the Workflow.
You can see the Exporter like special operators, that sees the whole
MSML file data structure and has dynamically input and output slots.

The creation of an Exporter begins with deriving from
\texttt{msml.exporter.Exporter} (figure~\ref{fig:exporteruml}). The
Exporter need to call \texttt{initialize} given:

\begin{itemize}
\item the MSML File data structure,
\item an exporter name,
\item supported features
\item logical and physical mesh sort and
\item a dictionary describing types of element parameters.
\end{itemize}

The initialization takes care of creating the appropriate input and
output slots. The supported features are  a subset of string.
If you set physical types for every input and output slot, you will
get the correct on access via \texttt{get\_value\_from\_memory()}.

The next step is \texttt{render()}. This method is
called first from the Executor. It should transfer the scene structure
into an appropriate representation for the simulation environment.
You should not modify the workflow memory or write absolute
filenames. Everything executed by the Executor has it's working
directory in the given output directory. The changes you want to make
should be memorized and returned by \texttt{execute()}.
There are two ways to access the referenced values from the workflow
memory: by calling \texttt{get\_value\_from\_memory()} with the correct
nodes from the scene, or by using the
\texttt{datamodel}. \texttt{datamodel} is a mirror data structure of scene,
where the referenced values are injected.

\texttt{execute()} is executed directly after \texttt{render()}, and
should execute the simulation and converts the results into
appropriates formats. The later step is only needed if there is no
automatically conversion defined.
Figure~\ref{fig:exporterskel} gives an sketch of an exporter.


% \begin{itemize}
% \item Exporter sees whole MSML File data structure
% \item Exporter traverse data structure, produce simulation files
% \item allowed to support a subset of features
% \item To phases \emph{render}, \emph{execute}
% \item definition of expected types (mesh, output request, parameter
%   for every element) (default: str)
% \item like operator with dynamically slots
% \item Data model
% \end{itemize}

\begin{figure}
  \centering
  \tikzstyle{tikzuml inherit style}=[color=black, -open triangle 45]

  \tikzumlset{fill class=white}
  \begin{tikzpicture}
    \umlclass[type=abstract]{Exporter}{%
      \# msml\_file : MSMLFile\\
      \# memory    : dict
      \# datamodel
    }{%
      + initialize(name, mesh\_type,\\
      \quad features, element\_types)\\
      + get\_value\_from\_memory(element)\\
      + \umlvirt{render() : void}\\
      + \umlvirt{execute() : dict}
    }
  \end{tikzpicture}
  \caption{UML Class Exporter}
  \label{fig:exporteruml}
\end{figure}

\begin{figure*}
  \centering
  \lstinputlisting[language=Python]{assets/exporterskel.py}
  \caption{Operator Definiton Example}
  \label{fig:exporterskel}
\end{figure*}


%%% Local Variables:
%%% mode: latex
%%% TeX-master: "../tutorial"
%%% End:
