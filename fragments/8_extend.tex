\section{Extend MSML}
\label{sec:extend}

MSML is a platform, that can be extended in multiple fashions.
The terms explain in section~\ref{sec:msml-architecture}

\subsection{Operator}
\label{sec:operator}

An operator in MSML is a function that perform an operation. At least
the function has to executable within Python. For C/C++ we provide a
CMAKE build environment and wrapping with SWIG. Additional we provide
operator adapters for external programs or shared object (ctype).

\begin{figure}
  \centering
\begin{lstlisting}[language=Python]
import vtk
def cp(mesh, ref):
  locator = vtk.vtkPointLocator()
  ugrid = read_ugrid(mesh)
  locator.SetDataSet(ugrid)

  index = locator.FindClosestPoint(ref)
  point = ugrid.GetPoint(index)
  distance = distance(ref, point)
  return {'index': index,
          'point': point,
           'dist': distance}
\end{lstlisting}
  \caption{Python snippet of a function for calculating the nearest
    point in \texttt{mesh} from \texttt{ref}}
  \label{fig:get_points}
\end{figure}

\begin{figure*}[h]
  \centering
  \lstinputlisting[language=XML]{assets/operatorexample.xml}
  \caption{Operator Definiton Example}
  \label{fig:operatorxml}
\end{figure*}


Let's assume we want to provide
\lstinline[language=Python]{cp(mesh, ref)} from
figure~\ref{fig:get_points} in MSML. The function calculates the
closest point in \lstinline[language=Python]{mesh} to a given
reference vector \lstinline[language=Python]{ref}.
Save the function in Python module, make sure MSML can import this
module by setting \lstinline[language=Python]{PYTHONPATH} or using the
\lstinline[language=Python]{--operator-dir} on the command line.
The next step is to create the entry in the
Alphabet. Figure~\ref{fig:operatorxml} shows an accuarate one.
The runtime gives the type of the operator. For a Python Operator you
need to specify the module and the function name. Input, output and
parameters contains list of args. The order of args determines the
order in which the arguments are given in the function call. Here we
define one input argument, that should be given as a VTK object, and
one parameter as a list of floats values. This operators delivers
three different output values.
Once you added the XML file (figure~\ref{fig:operatorxml}) to the
alphabet search path or add \lstinline{--alphabet-search-dir} on the
command line, you should be able to call this operator within MSML:
\begin{lstlisting}[language=XML,frame=none,numbers=none]
<closestPoint id="c"
 mesh="${mesh}"
 ref="2.2 3.5 6" />
\end{lstlisting}

You can access to every output with \lstinline{${c.distance}}, \lstinline{${c.point}}

\subsection{Element}
\label{sec:element}

Elements are very special. They describe information in the scene graph and are handled by the exporters. Normally any definition of a new element leads to a change in an exporter.

\begin{figure*}
  \centering
  \lstinputlisting[language=XML]{assets/elementexample.xml}
  \caption{Operator Definiton Example}
  \label{fig:operatorxml}
\end{figure*}


\subsection{Exporter}
\label{sec:exporter}


\begin{figure*}
  \centering
  \lstinputlisting[language=Python]{assets/exporterskel.py}
  \caption{Operator Definiton Example}
  \label{fig:operatorxml}
\end{figure*}


%%% Local Variables:
%%% mode: latex
%%% TeX-master: "../tutorial"
%%% End:
