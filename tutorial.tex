\documentclass[twocolumn,10pt,a4paper]{scrartcl}

\usepackage{xcolor}
\usepackage[T1]{fontenc} % Use 8-bit encoding that has 256 glyphs
\usepackage{microtype} % Slightly tweak font spacing for aesthetics
\usepackage[hang, small,labelfont=bf,up,textfont=it,up]{caption} % Custom captions under/above floats in tables or figures
\usepackage{booktabs} % Horizontal rules in tables
\usepackage{tikz-uml}
\usepackage{hyperref} % For hyperlinks in the PDF
\usepackage{paralist} % Used for the compactitem environment which makes bullet points with less space between them

\usepackage{kantlipsum}

\usepackage{abstract} % Allows abstract customization
\renewcommand{\abstractnamefont}{\normalfont\bfseries} % Set the "Abstract" text to bold
\renewcommand{\abstracttextfont}{\normalfont\small\itshape} % Set the abstract itself to small italic text

\usepackage{fancyhdr} % Headers and footers
\pagestyle{fancy} % All pages have headers and footers
\fancyhead{} % Blank out the default header
\fancyfoot{} % Blank out the default footer
%\fancyhead[C]{Running title $\bullet$ November 2012 $\bullet$ Vol. XXI, No. 1} % Custom header text
\fancyfoot[RO,LE]{\thepage} % Custom footer text

%
\usepackage{listings}
\lstset{% general command to set parameter(s)
  %basicstyle=\scriptsize, % print whole listing small
  keywordstyle=\color{black}\bfseries,
  % underlined bold black keywords
  identifierstyle=, % nothing happens
  commentstyle=\color{gray}, % white comments
  stringstyle=\ttfamily, % typewriter type for strings
  showstringspaces=true, % no special string spaces
  framexleftmargin=7mm,
  tabsize=2,
  showtabs=true,
  frame=single,
  rulesepcolor=\color{black},
  numbers=left,
  %linewidth=146mm,
  xleftmargin=8mm,
  basicstyle=\footnotesize\ttfamily,
  breaklines=true
}


%
%\usepackage{todonotes}
\newcommand\tdweigl[1]{\textcolor{purple!90}{\P\footnote{\emph{weigl:} \textcolor{purple}{#1}}}}

%-- tikz
\usepackage{tikz}
\usetikzlibrary{%
  positioning,
  automata,
  shapes,
  decorations,
  calc,
  shapes.multipart,
  arrows,
  chains,
  decorations.markings,
  intersections,
%  arrows.meta
}


%----------------------------------------------------------------------------------------

\begin{document}


%----------------------------------------------------------------------------------------
%	TITLE SECTION
%----------------------------------------------------------------------------------------

\title{Medical Simulation Markup Language}
\subtitle{A tutorial introduction}

\author{%
  Nicolai Schoch
  \thanks{\texttt{nicolai.schoch@iwr.uni-heidelberg.de}
    IWR EMCL}
  \and
  Markus Stoll
  \thanks{\texttt{m.stoll@dkfz-heidelberg.de}, DKFZ}
  \and
  Stefan Suwelack
  \thanks{\texttt{suwelack@kit.edu},
    Humanoids and Intelligence Lab, Karlsruher Institute for Technology}
  \and
  Alexander Weigl
  \thanks{\texttt{Alexander.Weigl@student.kit.edu},
    Humanoids and Intelligence Lab, Karlsruher Institute for Technology}
}

\date{\today}
%\thispagestyle{fancy} % All pages have headers and footers

\makeatletter
\twocolumn[
   \begin{@twocolumnfalse}
     \maketitle
     \begin{abstract}

		 Medical Simulation Markup Language defines....

		 	workflow, medical, simulation

			this tutorial shows example 
			how to use MSML
			and extend it

     \end{abstract}
    \end{@twocolumnfalse}
]
\makeatother
\saythanks

\section{Introduction}
\label{sec:introduction}

We first mentioned the idea from an
unified and mighty workflow system for biomechanical engineering  in
\cite{Suwelack:2014}. In the meantime we realize the Medical
Simulation Markup Language (MSML).

You can see MSML as a build tool, like \emph{GNU Make}, except the
offered operations are settled in the pre- and postprocessing of
three dimensional data after the biomechanical simulation.
The simulation is the heart of every workflow and is outsourced to
modern simulation frameworks, like
SOFA\footnote{http://www.sofa-framework.org/} or
Hiflow3\footnote{http://hiflow3.org}, and MSML cares about the messy
details of the simulation framework. You get an unified interface.

\paragraph{Content}
\label{sec:content}

In this paper we show the various power of MSML in different
scenarios (sections~\ref{sec:tut1,sec:tut2,sec:tut3}).
Before the tutorials we create a view on MSML in
section~\ref{sec:msml-architecture}. This chapter defines the ground
terms and nomenclatures in MSML. Openness is one power of MSML. You
can extend it in various kinds, such topics are discovered in
section~\ref{sec:extend}.

\paragraph*{Acknowledgement}
\label{sec:acknowledgement}

MSML was developed in the SFB\,TRR\,125 Cognition Guided Surgery
project, funded by the DFG.\tdweigl{add english name here}


%%% Local Variables:
%%% mode: latex
%%% TeX-master: "../tutorial"
%%% End:

\section{MSML Architecture}
\label{sec:msml-architecture}

We start with the terms, give an overview of the pipeline and end this
section with the MSMLFile datastructure.

\paragraph{Terms}

An \emph{Operator} is a function, that can be used within the
workflow. The function can be written in C/C++ or Python. We maintain additional
meta data of the arguments and output in a separate XML
file. If an Operator is used, it get instantiated with the given
references to other instances or variables. We call instances of
Operator a \emph{Task}. \emph{Elements} define additional features,
like materials or constraints, to your objects in the scene.
The \emph{Alphabet} contains both, Elements and Operators, with their
XML meta data. You can add new Operators or Elements to MSML by
creating new Alphabet entries. The simulation is outsourced to several
simulation frameworks. The interface between MSML and the frameworks
is done within several specific \emph{Exporter}. Currently we support
Sofa, Hiflow, FeBio and Abaqus.

\paragraph{Pipeline}
\label{sec:pipeline}

Figure~\ref{fig:pipeline} gives an overview about the process.
MSML takes a data structure as shown in figure~\ref{fig:model}.
The data structure can allocated from XML or with Python.
Defining in Python offers more control over the pipeline and is more
flexible, as you can use parameterized the creation.
XML is easy, understandable but more fixed.

The first step analyze the data structure for finding errors and
creating a build graph. The build graph is a directed acyclic
graph\footnote{it has to be},
\begin{figure}
  \centering
  \tikzset{
    pp state/.style = {draw, black, text width=7em, text height=1.2em,
      rounded corners, text centered, text depth=.25ex},
    pp line/.style = {->,draw, line width=0.8 pt},
    pp border/.style = {dotted}
  }

  \begin{tikzpicture}[node distance=0.5cm]

    \node[pp state] (model) {MSML Model};

    \node[pp state, above right=0.6cm and -1.4cm of model] (XML) {XML file};
    \node[pp state, above left= 0.6cm and -1.4cm  of model] (py) {Python};

    \node[pp state, below=of model] (bg)    {Build Graph};

    \node[pp state, below=of bg] (pre) {Preprocessing};
    \node[pp state, below=of pre] (sim) {Simulation};
    \node[pp state, below=of sim] (post) {Postprocessing};

    \path[pp line]
      (py) edge (model)
      (XML) edge (model)
      (model) edge (bg)
      (bg) edge (pre)
      (pre) edge (sim)
      (sim) edge (post)
    ;


    \draw[pp border]
      ($(pre.north west) + (-0.3,0.3)$) rectangle
      ($(post.south east) + (+0.3,-0.3)$)
    ;

  \end{tikzpicture}

  \caption{MSML Pipeline}
  \label{fig:pipeline}
\end{figure}



\paragraph{Data structure}
\label{sec:data-structure}


The MSMLFile gather all information for making a simulation. It
contains the Workflow, that contains the Task to be executed, the
environment with parameters for the simulation framework, variables,
and scene objects. Every scene object is a node in the simulation and
have constraints, material regions or output.





\begin{figure*}
  \centering
  \tikzset{
    uml class/.style = {draw, black, minimum width = 8em, minimum
      height = 2em, text centered, text depth=.25ex, rectangle},
    uml class 3/.style = {draw, black, minimum width = 8em, minimum
      height = 2em, text centered, text depth=.25ex, rectangle split,
      rectangle split parts=3},
    uml comp/.style ={>=diamond}
  }
  \begin{tikzpicture}
    %\draw[ultra thin,gray] (-10,-10) grid (10,10);

    \node[uml class] (MF) {MSMLFile};

    \node[uml class,left=of MF] (SO) {SceneObject};

    \node[uml class,right=of MF] (Env) {Environment};

    \node[uml class,above=of MF] (WF) {Workflow};

    \node[uml class,below=of MF] (V) {Variables};

    \node[uml class,above=of WF] (T) {Task};

    \node[uml class,above left=of SO] (C) {Constraints};
    \node[uml class,left=of SO] (MR) {Material Region};
    \node[uml class,below left=of SO] (OR) {Output};

    \newcommand\compositon[2]{
      \draw [{Diamond}-{Stealth}] #1 -- #2;
    }

    %\draw[{Diamond}-{Stealth}] (V) --
    %  node[near end, right]   {1}
    %  node[near start, right] {*}
    %  (MF);


    \umlunicompo[mult=*]{MF}{V}
    \umluniassoc[anchors=east and west]{MF}{Env}
    \umluniassoc{MF}{WF}
    \umluniassoc[mult1=1,mult2=*]{MF}{SO}

    \umlunicompo[mult=]{SO}{C}
    \umlunicompo[mult=]{SO}{OR}
    \umlunicompo[mult=]{SO}{MR}

    \umlunicompo[mult=*]{WF}{T}

  \end{tikzpicture}
  \caption{MSML Model}
  \label{fig:model}
\end{figure*}

%%% Local Variables:
%%% mode: pdflatex
%%% TeX-master: "../tutorial"
%%% End:


\section{Standford Bunny}
\label{sec:bunny}

\begin{itemize}
\item explain MSML structure
\item show simulation sequence (maybe with transparent overlay)
\item little timings
\end{itemize}

\kant[10]


\kant[10]


\kant[10]


\kant[10]


\kant[10]


\kant[10]


\kant[10]


\kant[10]


\kant[10]


\kant[10]


\kant[10]


\kant[10]

\section{Lungs}
\label{sec:lung}



\kant[10]


\kant[10]


\kant[10]


\kant[10]


\kant[10]


\kant[10]


\kant[10]


\kant[10]


\kant[10]


\kant[10]


\kant[10]


\kant[10]


\kant[10]


\kant[10]


\kant[10]


\kant[10]


\kant[10]


\kant[10]


\kant[10]


\kant[10]


\kant[10]


\kant[10]


\kant[10]


\kant[10]


\kant[10]


\kant[10]


\kant[10]


\kant[10]


\kant[10]


\kant[10]


\kant[10]


\kant[10]


\kant[10]


\kant[10]


\kant[10]


\kant[10]


\kant[10]


\kant[10]


\kant[10]


\kant[10]


\kant[10]


\kant[10]


\kant[10]


\kant[10]


\kant[10]


\kant[10]


\kant[10]


\kant[10]
\kant[5]


\section{Lungs}
\label{sec:lung}


\section{Mitral Valves}
\label{sec:mitral}




\section{Extend MSML}
\label{sec:extend}

MSML is a platform, that can be extended in multiple fashions.
The terms explain in section~\ref{sec:msml-architecture}

\subsection{Operator}
\label{sec:operator}

An operator in MSML is a function that perform an operation. At least
the function has to executable within Python. For C/C++ we provide a
CMAKE build environment and wrapping with SWIG. Additional we provide
operator adapters for external programs or shared object (ctype).

\begin{figure}
  \centering
\begin{lstlisting}[language=Python]
import vtk
def cp(mesh, ref):
  locator = vtk.vtkPointLocator()
  ugrid = read_ugrid(mesh)
  locator.SetDataSet(ugrid)

  index = locator.FindClosestPoint(ref)
  point = ugrid.GetPoint(index)
  distance = distance(ref, point)
  return {'index': index,
          'point': point,
           'dist': distance}
\end{lstlisting}
  \caption{Python snippet of a function for calculating the nearest
    point in \texttt{mesh} from \texttt{ref}}
  \label{fig:get_points}
\end{figure}

\begin{figure*}[h]
  \centering
  \lstinputlisting[language=XML]{assets/operatorexample.xml}
  \caption{Operator Definiton Example}
  \label{fig:operatorxml}
\end{figure*}


Let's assume we want to provide
\lstinline[language=Python]{cp(mesh, ref)} from
figure~\ref{fig:get_points} in MSML. The function calculates the
closest point in \lstinline[language=Python]{mesh} to a given
reference vector \lstinline[language=Python]{ref}.
Save the function in Python module, make sure MSML can import this
module by setting \lstinline[language=Python]{PYTHONPATH} or using the
\lstinline[language=Python]{--operator-dir} on the command line.
The next step is to create the entry in the
Alphabet. Figure~\ref{fig:operatorxml} shows an accuarate one.
The runtime gives the type of the operator. For a Python Operator you
need to specify the module and the function name. Input, output and
parameters contains list of args. The order of args determines the
order in which the arguments are given in the function call. Here we
define one input argument, that should be given as a VTK object, and
one parameter as a list of floats values. This operators delivers
three different output values.
Once you added the XML file (figure~\ref{fig:operatorxml}) to the
alphabet search path or add \lstinline{--alphabet-search-dir} on the
command line, you should be able to call this operator within MSML:
\begin{lstlisting}[language=XML,frame=none,numbers=none]
<closestPoint id="c"
 mesh="${mesh}"
 ref="2.2 3.5 6" />
\end{lstlisting}

You can access to every output with \lstinline{${c.distance}}, \lstinline{${c.point}}

\subsection{Element}
\label{sec:element}

Elements are very special. They describe information in the scene graph and are handled by the exporters. Normally any definition of a new element leads to a change in an exporter.

\begin{figure*}
  \centering
  \lstinputlisting[language=XML]{assets/elementexample.xml}
  \caption{Operator Definiton Example}
  \label{fig:operatorxml}
\end{figure*}


\subsection{Exporter}
\label{sec:exporter}


\begin{figure*}
  \centering
  \lstinputlisting[language=Python]{assets/exporterskel.py}
  \caption{Operator Definiton Example}
  \label{fig:operatorxml}
\end{figure*}


%%% Local Variables:
%%% mode: latex
%%% TeX-master: "../tutorial"
%%% End:

\section{Conclusion}
\label{sec:conclusion}

We hope to give you an introduction in the world of MSML.
MSML is more than just a build or workflow tool. The project offers
easily to use Python and C++ function.

MSML is driven towards an semantic an intelligent system.

\tdweigl{Mailinglist Hint}




\bibliographystyle{plain}
\bibliography{refs.bib}

\end{document}

%%% Local Variables:
%%% mode: pdflatex
%%% TeX-master: t
%%% End:
